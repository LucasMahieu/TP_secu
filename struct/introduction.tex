De nos jours, le chiffrement de données est essentiel pour assurer la sécurité
des transferts de données et préserver la sécurité d'un pays, d'une entreprise
ou la vie privée d'un individu.

L'algorithme AES (Advanced Encryption Standard) est un algorithme très utilisé
en chiffrement des données car réputé incassable. Néanmoins, nous allons voir,
au travers de deux exemples réalisés en travaux pratiques, que cette sécurité
dépend grandement de l'implémentation qu'il en est fait dans un circuit
électronique.\\

Dans un premier temps, nous allons réaliser une attaque par injection de fautes sur un
circuit AES utilisant une protection par redondance d'informations.
Ensuite nous réaliserons une attaque par canaux auxiliaires (Differential Power Analysis) sur un circuit
AES n'ayant aucune protection spéciale.
