\subsection{Quelle protections sont mises en place contre les attaques par fautes ?}

Bien que l'algorithme de l'AES est incassable aujourd'hui, son implementation peut permettre de récupérer des données sensibles par de nombreux moyens.
Dans ce TP, on s'intéresse à une potentielle attaque par injection de fautes et a une contre-mesure de cette attaque. 
Pour mener un attaque par faute  dans ce module AES, nous avons tout d'abord analysé le circuit et tenté de comprendre comment le système 

\subsection{Quelles sont les faiblesses de cette sécurité ?}

\subsection{Implémentation de l'attaque par faute}

En utilisant la faiblesse expliquée précedement, notre implémentation
consiste à introduire une faute de retournement de bits sur le port
\texttt{in_hi}, sur 8 bits, de la colonne 0. Nous nous basons sur un
modèle réaliste d'injection de faute où, grâce à un laser, l'attaquant
pourrait retourner les 8 bits de ce port.\\
\texttt{
  in_hi_sig <= in_hi; \\
  lin_0_in <= in_hi_sig XOR fault_sig;
} \\
Le signal \texttt{fault_sig} est remonté jusque dans le fichier top, comme un
port du composant \texttt{aes_core}, pour pouvoir générer l'attaque.
Grâce au XOR,
un bit à 1 dans le signal \texttt{fault_sig} retournera le bit correspondant
dans le signal \texttt{in_hi} et créera une faute.
Précisons que nous n'attaquons que la première colonne du DataUnit et que ce
même signal \texttt{fault_sig} est laissé à \texttt{"00000000"} pour les trois
autres colonnes. \\
Nous avons tout d'abord testé cette implémentation en modifiant le code
VHDL qui instancie notre IP.


% Utilisation du signal fault_sig pour déterminer quels bits
% seront flippés (avec un XOR)
% Utilisation du fichier user_logic pour écrire le code de test

\subsection{Embarcation du code}

% Code en C pour contrôler l'IP à travers les registres, processeur MicroBlaze
% Simulation après synthese sur processeur MicroBlaze non fonctionelle, faute
% détéctée.

\subsection{Conclusion}
