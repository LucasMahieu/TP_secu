\subsection{Introduction}

Dans cette partie nous allons effectuer une attaque par canaux auxiliaires
sur un chiffrement AES. Cette attaque s'appelle Differential Power Analysis ou
DPA, et consiste à étudier la consommation électrique du système visé sur de 
multiples exécutions et d'en déduire par des procédés statistiques l'information
visée, ici la clé de chiffrement.

\subsection{Théorie}
La sécurité d'un système de chiffrement dépend de l'algorithme employé et de 
son implémentation sur circuit électronique. Dans le cas de la DPA, nous allons 
utiliser des faiblesses d'implémentations pour en déduire la clé de chiffrement 
employée. 

La SBox (Substitution Box) est le composant non linéaire principal de l'AES. Il
s'agit d'une substitution d'octets, cette opération est effectuée juste après
l'ajout de la clé à la donnée que l'on veut chiffrer. Il est donc possible,
à partir du résultat de la première itération de l'AES qui ne dépendent que 
des données d'entrées et de la clé, de retrouver la clé.

La DPA est une attaque par canaux cachés qui utilisent la consommation
électrique du circuit pour valider ou invalider des hyptohèses sur la clé.
La consommation d'un circuit logique dépend grandement des données traitées.
Dans notre cas, il n'y a pas de contre-mesure spécifique utilisée pour
équilibrer la consommation. Ainsi, nous devrions être capables de réaliser
une attaque DPA sur la SBox de l'AES.

\subsection{Travail réalisé}

Tout d'abord, nous devons transformer les fichiers VHDL qui nous ont été donnés
en une description au niveau des transistors, pour pouvoir en extraire une
consommation électrique simulée. Cette étape est réalisée à l'aide de
l'environnement Cadence.

INSERER IMAGES 

Une fois cette description générée, nous utilisons le simulateur de consommation
électrique Synopsis Nanosim pour obtenir les profils de courants selon certains 
stimulis.

INSERER IMAGES

Le logiciel d'analyse DPA fourni nous permet de retrouver les clés utilisées.
Il accepte deux paramètres : l'index du bit attaqué et le chemin vers le
dossier contenant les fichiers de simulation.
L'outil lit un fichier de configuration contenant une liste des vecteurs qui
seront utilisés pour la DPA. Ensuite, les données simulées sont collectées à 
partir des fichiers de simulation. Pour chaque hypothèse de clé, il partitionne
les traces de simulation de chaque message selon la valeur du bit attaqué.
Enfin, il évalue les différences entre les moyennes de chaque partition et 
sélectionne la valeur la plus haute.


\subsection{Résultats}

INSERER IMAGES ET DIRE QUE C'EST BIEN, QU'ON A REUSSI. YOUPI.
TPCLIC.
